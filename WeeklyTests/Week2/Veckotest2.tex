\documentclass{article}
\usepackage{amsmath}
\usepackage[utf8]{inputenc}
\usepackage{graphicx}
\usepackage{enumerate}
\setlength{\parskip}{\baselineskip}%
\setlength{\parindent}{0pt}%

\begin{document}
\title{Lösning Veckotest 2, MA1439}
\author{Henrik Samuelsson}
\maketitle
\clearpage

\section*{Uppgift 1.} 
Låt $p(x)=1-7x-2x^2$. Beräkna

\begin{enumerate}[(a)]
\item $p(1)$
\item $p(-2)$
\end{enumerate}

\textbf{Lösning}

\begin{enumerate}[(a)]
\item $p(1) = 1 - 7 \cdot 1 - 2 \cdot 1^2 = 1 - 7 - 2 \cdot 1 = 1 - 7 - 2 = -8$
\item $p(-2)= 1 - 7 \cdot (-2) - 2 \cdot (-2^2) = 1 + 14 - 2\cdot 4 = 1 + 14 - 8 = 7$
\end{enumerate}

\section*{Uppgift 2.} 
Förenkla $3x(2-x)-4(x^2-x+3)$.

\textbf{Lösning}

$3x(2-x)-4(x^2-x+3)$

$6x-3x^2-4x^2+4x-12$

$-7x^2+10x-12$

\section*{Uppgift 3.} 
Lös ekvationen $(x+5)^2=(x-3)(x+5)+6(10+x)$.

\textbf{Lösning}

$(x+5)^2=(x-3)(x+5)+6(10+x)$

$x^2+10x+25=x^2+5x-3x-15+60+6x$

$x^2+10x+25=x^2+8x+45$

$2x=20$

$x=10$

\section*{Uppgift 4.} 
Dela upp i faktorer så långt så möjligt
\begin{enumerate}[(a)]
\item $49v^2-4u^2$
\item $64x^2-48x+9$
\end{enumerate}

\textbf{Lösning}

Konjugat- och kvadrerings-regeln ger
\begin{enumerate}[(a)]
\item $49v^2-4u^2=(7v+2u)(7v-2u)$
\item $64x^2-48x+9=(8x-3)(8x-3)$
\end{enumerate}

\section*{Uppgift 5.}
I en rätvinklig triangel är hypotenusan 7 cm längre än den längsta kateten. Den kortaste kateten är 21cm. Beräkna triangelns omkrets.

\textbf{Lösning}

Vi saknar längden av den ena kateten men eftersom triangeln är rätvinklig så kan vi fram den med hjälp av Pythagoras sats

$(k+7)^2=21^2+k^2$

$k^2+14k+49=441+k^2$

$14k=392$

$k=28$

Triangelns omkrets ges av längden av hypotenusan plus längden av de båda kateterna

(28+7)+28+21=84
\section*{Uppgift 5.}
Lös ekvationssystemet nedan med valfri metod

$\begin{cases}
3x-y-2=0\\
3x+2y-10=0
\end{cases}$

\textbf{Lösning}

Vi börjar med att lösa ut $y$ ur första ekvationen och sätter resultatet i andra ekvationen

$\begin{cases}
y=3x-2\\
3x+2(3x-2)-10=0
\end{cases}$

Lös sedan ut $x$ ur andra ekvationen

$\begin{cases}
y=3x-2\\
3x+6x-4-10=0
\end{cases}$

$\begin{cases}
y=3x-2\\
9x=14
\end{cases}$

$\begin{cases}
y=3x-2\\
x=14/9
\end{cases}$

Vi vet värdet av $x$ nu. Återstår bara att använda detta värdet i första ekvationen för att få reda på vad $y$ är

$\begin{cases}
y=3\cdot14/9-2\\
x=14/9
\end{cases}$

$\begin{cases}
y=8/3\\
x=14/9
\end{cases}$

\section*{Uppgift 6.}
Lös följande ekvationssystem med additionsmetoden

$\begin{cases}
3x+2y=1\\
2x-y=-11
\end{cases}$

\textbf{Lösning}
Vi noterar att om vi multiplicerar den andra ekvationen med $2$ så blir koefficienten framför $y$ motsatta tal. Vi väljer att göra detta

$\begin{cases}
3x+2y=1\\
4x-2y=-22
\end{cases}$

Addera nu de båda ekvationerna ledvis vilket kommer att eliminera $y$

$3x+4x+2y-2y=1-22$

$7x=-21$

$x=-3$

Vi kan nu ta reda på $y$ genom att sätta in värdet av $x$ i någon av de två originalekvationerna

$3\cdot(-3)+2y=1$

$-9+2y=1$

$y=5$
\end{document}
