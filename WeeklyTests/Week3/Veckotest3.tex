\documentclass{article}
\usepackage{amsmath}
\usepackage[utf8]{inputenc}
\usepackage{graphicx}
\usepackage{enumerate}
\setlength{\parskip}{\baselineskip}%
\setlength{\parindent}{0pt}%

\begin{document}
\title{Veckotest 3, MA1439}
\author{Henrik Samuelsson}
\maketitle

\section*{Uppgift 1.}
Lös andragradsekvationerna
\begin{enumerate}[(a)]
\item $x^2+8x+7=0$
\item $x(x+23)=0$
\item $2x^2-8x-10=0$
\end{enumerate}

\textbf{Lösning}
\begin{enumerate}[(a)]
\item
$x=-\dfrac{8}{2}\pm\sqrt{\left(\dfrac{8}{2}\right)^2-7}=-4\pm\sqrt{16-7}=-4\pm3$

$x_1 = -1$

$x_2 = -7$
\item
Denna uppgift är så pass trivial att man kan se direkt att lösningen är\\
$x_1 = 0$

$x_2 = -23$
\end{enumerate}
\end{document}
