\documentclass{article}
\usepackage[utf8]{inputenc}
\usepackage{graphicx}
\setlength{\parskip}{\baselineskip}%
\setlength{\parindent}{0pt}%

\begin{document}
\textbf{1.} Lös ekvationen $7x-2(3x-8)=45$

\textbf{Lösning}\\
$7x-2(3x-8)=45$\\
$7x-6x+16=45$\\
$7x-6x=45-16$\\
$x=29$

\textbf{2.} Låt $f(x)=4x^{2}+2x$ och bestäm\\
\textbf{a)} $f(0)$\\
\textbf{b)} $f(3)$\\
\textbf{c)} $f(-1)$\\
\textbf{d)} $f(2a + 3a)$

\textbf{Lösning}\\
\textbf{a)} $f(0)=4\cdot0^2+2\cdot0=4\cdot0=0$\\
\textbf{b)} $f(3)=4\cdot3^2+2\cdot3=4\cdot9+4=36+4=40$\\
\textbf{c)} $f(-1)=4(-1)^2+2(-1)=4\cdot1-2=4-2=2$\\
\textbf{d)} $f(2a+3a)=4\cdot(2a+3a)^2+2\cdot(2a+3a)=4(4a^2+12a+9a^2)+10a=52a^2+58a$

\textbf{3.} En linje är parallell med linjen $2x - y = 0$ och går genom punkten $(4, -1)$. Bestäm linjens ekvation.

\textbf{Lösning}\\
Den kända linjen kan skrivas som $y = 2x$. Den sökta linjen kan skrivas på formen $y = 2x + m$ då den är parallell med den kända linjen. Vi kan få fram värdet av $m$ genom att sätta in koordinaterna för den kända punkten i linjens ekvation.

$-1 = 2 \cdot 4 + m$\\
$-1 = 8 + m$\\
$m = -1 - 8$\\
$m = -9$

Den sökta linjens ekvation är således $y = 2x - 9$.

I bilden nedan ser vi grafer av dom två linjerna. Resultatet verkar rimligt.

\includegraphics[scale=0.65]{graph_1_3_1.png} 

\end{document}
