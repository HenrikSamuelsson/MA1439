\documentclass{article}
\usepackage{amsmath}
\usepackage[utf8]{inputenc}
\usepackage{graphicx}
\usepackage{enumerate}
\setlength{\parskip}{\baselineskip}%
\setlength{\parindent}{0pt}%

\begin{document}

\begin{titlepage}
\title{Lösning av Inlämningsuppgift \\ Matematik 1 för produktutveckling, MA1439 \\ Lp 2, Ht 15}
\author{Henrik Samuelsson}
\maketitle
\thispagestyle{empty}
\end{titlepage}

\section*{Uppgift 1.}
Förenkla nedanstående uttryck så långt det går

$\dfrac{x^2 - 9}{x - 1} \cdot \dfrac{x^2 - 2x + 1}{x + 3}$

\textbf{Lösning}

Börja med skriva om uttrycket genom att  faktorisera de båda täljarna. Vi använder konjugatregeln och en av kvadreringsreglerna när vi faktoriserar.

$\dfrac{(x + 3)(x-3)}{x - 1} \cdot \dfrac{(x-1)^2}{x + 3}$

Det går nu att eliminera båda nämnarna i uttrycket genom att stryka mot motsvarande termer i täljarna. Vi får kvar följande förenklade uttryck. 

$(x-3)(x-1)$

\textbf{Svar}

Det förenklade uttrycket blir $(x-3)(x-1)$

\end{document}
